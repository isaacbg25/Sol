\documentclass[11pt]{article}
\usepackage{graphicx} % Required for inserting images
\usepackage[top=2.5cm, bottom=2.5cm, left=2.5cm, right=2.5cm]{geometry}
\usepackage[T1]{fontenc}
\usepackage{hyperref}
\usepackage[utf8]{inputenc}
\usepackage{multirow}
\usepackage{subcaption}
\usepackage{booktabs}
\usepackage{bookmark}
\usepackage{graphicx}
\usepackage{setspace}
\setlength{\parindent}{0in}
\usepackage{physics}
\usepackage{tikz}
\usepackage{tikz-3dplot}
\usepackage[outline]{contour} % glow around text
\usepackage{xcolor}
\usepackage{float}
\usepackage{makeidx}
\usepackage{fancyhdr}
\usepackage{pgfplots}
\usepackage{amsmath}
\pgfplotsset{compat=1.18}
\usepackage{caption}
\usepackage[english,catalan]{babel}
\setlength{\parskip}{11pt}
\usepackage{xcolor}
\usepackage{listings}
\usepackage{marginnote}

\title{\Huge\bfseries Pràctica de simulació: \\ Instal·lació de panells solars fotovoltaics en un habitatge unifamiliar a Catalunya \\ [2ex] \Large}

\author{\begin{tabular}{c}
\textbf{GRUP C3} \\
Isaac Baldi García (1667260)\\
Marcel López Freixes (1668323) \\
Eira Jacas García (1666616) \\
Núria Castillo Ariño (1669145)
\end{tabular}}

\date{07/1/01/2025}

\begin{document}

\maketitle

Abstract pràctica 

\section{Moviment de la Terra al voltant del Sol} 
En aquesta secció ens hem proposat simular el moviment de translació de la Terra al voltant del Sol. Per fer-ho hem partit de la Llei de la Gravitació Univeral i hem simplificant el nostre problema de dos cossos a un d'un sol cos sota una força central, $F(r)$.
\begin{equation}
    F(r)=-\frac{GMm}{r^2}
\end{equation}

On G és la constant de grabitació universal, M la massa del Sol i m la massa de la Terra.\footnote{Totes les dades orbitalàries agafades del Jet Propulsion Laboratory de la NASA: \url{https://ssd.jpl.nasa.gov/}}

Per aquest tipus de sistemes i considerant únicament aquesta força central, tenim dues equacions de moviment en el pla polar 
\begin{equation}
    F(r)=m\ddot{r}-mr{\dot{\theta}}^2
    \label{equ_en_r}
\end{equation}
\begin{equation}
    0=\ddot{\theta}m=mr\ddot{\theta}+2m\dot{r}\dot{\theta}
    \label{equ_en_theta}
\end{equation}
i la propietat que el moment angular es conserva
\begin{equation}
    L=mr\dot{\theta}=ctt
    \label{moment_angular}
\end{equation}

Combinant les equacions \eqref{equ_en_r} i \eqref{moment_angular} obtenim una EDO que només depen de r i una EDO que només depen de $\theta$
\begin{equation}
    \frac{\partial\dot{r}}{\partial t}=-GM\frac{1}{r^2}+\frac{L^2}{m^2r^3}
    \label{edor}
\end{equation}
\begin{equation}
    \frac{\partial}{}
    \label{edot}
\end{equation}
Normalitzant aquestes dues equacions i reduint l'ordre de l'equació \eqref{edor} obtenim
\begin{equation}
    \frac{\partial\tilde{v}}{\partial\tilde{t}}=-\frac{1}{\tilde{r}^2}+\frac{1}{\tilde{r}^3}
    \label{1_edo_r}
\end{equation}
\begin{equation}
    \frac{\partial\tilde{r}}{\partial\tilde{t}}=\tilde{v}
    \label{2_edo_r}
\end{equation}
\begin{equation}
    \frac{\partial\tilde{\theta}}{\partial\tilde{t}}=\frac{1}{\tilde{r}^2}
    \label{edo_tetha}
\end{equation}
on les variables normalitzades segueixen $r=\tilde{r}\alpha$, $t=\tilde{t}\frac{\alpha}{\bar{v}}$, $v=\tilde{v}\bar{v}$ i les constants de normalització $\alpha = \frac{\beta}{\kappa}$, $\bar{v}=\frac{\kappa}{(\beta)^{1/2}}$, $\beta=\frac{L^2}{m^2}, \kappa=GM$.

Aquest sistema d'equacions diferencials de primer ordre l'hem resolt numèricament amb el mètode d'Euler i agafant com a condicions de contorn el radi de l'òrbita, la velocitat radial i l'angle al periheli.\footnotemark[\value{footnote}]
Si grafiquem els resultats obtenim
\begin{figure}[h]
    \centering
    \includegraphics[width=0.5\textwidth]{orbita.png}
    \label{orb_terra}
    \caption{Òrbita Terrestre calculada numèricament}
\end{figure}

El mètode d'Euler és poc exacte però amb una discretització prou fina dona bons resultats. Si calculem l'error acomulat en el radi al completar una Òrbita sencera amb una discretització temporal de $3000$ punts obtenim
\begin{equation}
    Error = \frac{1.47098075136e11-1.47082017410e11}{1.47098075136e11}100\approx0,01\%
\end{equation}
Tot i així a la secció \ref{sec: edos} resoldrem aquest sistema d'EDOs amb altres mètodes per comparar-ne els resultats.


\section{Posició del Sol al cel vist des de l'habitatge} 
En aquesta secció ens proposem trobar la posició del Sol des d'un punt determinat de la Terra durant tot l'any.\footnote{Per fer els càlculs hem agafat les coordenades d'un habitatge de Sant Cugat del Vallès} Per fer-ho hem parametritzat la posició del sol amb dos angles, l'angle vertical $\nu$ i l'angle horitzontal $\eta$ (\ref{fig: sist_sol}), i hem definit els vectors $\vec{\rho}$, $\vec{r}$ i $\vec{R}$ (\ref{fig: sist_vectors}).
També hem definit tres sistemes de referència per facilitar els càlculs i trobar els àngles esmentats:
\begin{itemize}
    \item sistema $\Omega$: 
    \item sistema $\beta$: inclou la inclinacó de l'eix de rotació
    \item sistema $\gamma$: inclou la rotació respecte el periheli i l solstici d'hivern
\end{itemize}
Al sistema $\omega$ el vector r queda
... 

El vector R ja el tenim de la seccio ... 

el vector rho és simplement la suma 

Un cop tenim aquests tres vectors, per trobar els  angles només hem de calcular... 

\section{Estudi de l'energia elèctrica}

\section{Resolució de l'EDO per diversos mètodes numèrics}\label{sec: edos}

\subsection{Resolució de l'EDO: Runge-Kutta d'ordre 2}

\subsection{Resolució de l'EDO: Runge-Kutta d'ordre 4}


\end{document}