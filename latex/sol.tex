\documentclass[11pt]{article}
\usepackage{graphicx} % Required for inserting images
\usepackage[top=2.5cm, bottom=2.5cm, left=2.5cm, right=2.5cm]{geometry}
\usepackage[T1]{fontenc}
\usepackage{hyperref}
\usepackage[utf8]{inputenc}
\usepackage{multirow}
\usepackage{subcaption}
\usepackage{booktabs}
\usepackage{bookmark}
\usepackage{graphicx}
\usepackage{setspace}
\setlength{\parindent}{0in}
\usepackage{physics}
\usepackage{tikz}
\usepackage{tikz-3dplot}
\usepackage[outline]{contour} % glow around text
\usepackage{xcolor}
\usepackage{float}
\usepackage{makeidx}
\usepackage{fancyhdr}
\usepackage{pgfplots}
\usepackage{amsmath}
\pgfplotsset{compat=1.18}
\usepackage{caption}
\usepackage[english,catalan]{babel}
\setlength{\parskip}{11pt}
\usepackage{xcolor}
\usepackage{listings}
\usepackage{marginnote}

\title{\Huge\bfseries Pràctica de simulació: \\ Instal·lació de panells solars fotovoltaics en un habitatge unifamiliar a Catalunya \\ [2ex] \Large}

\author{\begin{tabular}{c}
\textbf{GRUP C3} \\
Isaac Baldi García (1667260)\\
Marcel López Freixes (1668323) \\
Eira Jacas García (1666616) \\
Núria Castillo Ariño (1669145)
\end{tabular}}

\date{07/01/2025}

\begin{document}

\maketitle
\newpage

\tableofcontents
\newpage

\section{Moviment de la Terra al voltant del Sol} \label{sec: seccio_1}
En aquesta secció ens hem proposat simular el moviment de translació de la Terra al voltant del Sol. Per fer-ho hem partit de la Llei de la Gravitació Univeral i hem simplificant el nostre problema de dos cossos a un d'un sol cos sota una força central, $F(r)$.
\begin{equation}
    F(r)=-\frac{GMm}{r^2}
\end{equation}

On G és la constant de grabitació universal, M la massa del Sol i m la massa de la Terra.\footnote{Totes les dades orbitalàries agafades del Jet Propulsion Laboratory de la NASA: \url{https://ssd.jpl.nasa.gov/}}

Per aquest tipus de sistemes i considerant únicament aquesta força central, tenim dues equacions de moviment en el pla polar 
\begin{equation}
    F(r)=m\ddot{r}-mr{\dot{\theta}}^2
    \label{equ_en_r}
\end{equation}
\begin{equation}
    0=\ddot{\theta}m=mr\ddot{\theta}+2m\dot{r}\dot{\theta}
    \label{equ_en_theta}
\end{equation}
i la propietat que el moment angular es conserva
\begin{equation}
    L=mr\dot{\theta}=ctt
    \label{moment_angular}
\end{equation}

Combinant les equacions \eqref{equ_en_r} i \eqref{moment_angular} obtenim una EDO que només depen de r i una EDO que només depen de $\theta$
\begin{equation}
    \frac{\partial\dot{r}}{\partial t}=-GM\frac{1}{r^2}+\frac{L^2}{m^2r^3}
    \label{edor}
\end{equation}
\begin{equation}
    \frac{\partial\theta}{\partial t}=\frac{L}{mr^2}
    \label{edot}
\end{equation}
Normalitzant aquestes dues equacions i reduint l'ordre de l'equació \eqref{edor} obtenim
\begin{equation}
    \frac{\partial\tilde{v}}{\partial\tilde{t}}=-\frac{1}{\tilde{r}^2}+\frac{1}{\tilde{r}^3}
    \label{1_edo_r}
\end{equation}
\begin{equation}
    \frac{\partial\tilde{r}}{\partial\tilde{t}}=\tilde{v}
    \label{2_edo_r}
\end{equation}
\begin{equation}
    \frac{\partial\tilde{\theta}}{\partial\tilde{t}}=\frac{1}{\tilde{r}^2}
    \label{edo_tetha}
\end{equation}
on les variables normalitzades segueixen $r=\tilde{r}\alpha$, $t=\tilde{t}\frac{\alpha}{\bar{v}}$, $v=\tilde{v}\bar{v}$ i les constants de normalització $\alpha = \frac{\beta}{\kappa}$, $\bar{v}=\frac{\kappa}{(\beta)^{1/2}}$, $\beta=\frac{L^2}{m^2}, \kappa=GM$.

Aquest sistema d'equacions diferencials de primer ordre l'hem resolt numèricament amb el mètode d'Euler i agafant com a condicions de contorn el radi de l'òrbita, la velocitat radial i l'angle al periheli.\footnotemark[\value{footnote}]
Si grafiquem els resultats obtenim
\begin{figure}[h]
    \centering
    \includegraphics[width=0.5\textwidth]{orbita.png}
    \label{orb_terra}
    \caption{Òrbita Terrestre calculada numèricament}
\end{figure}

El mètode d'Euler és poc exacte però amb una discretització prou fina dona bons resultats. Si calculem l'error acomulat en el radi al completar una Òrbita sencera amb una discretització temporal de $3000$ punts obtenim
\begin{equation}
    Error = \frac{1.47098075136e11-1.47082017410e11}{1.47098075136e11}100\approx0,01\%
\end{equation}
Tot i així a la secció \ref{sec: edos} resoldrem aquest sistema d'EDOs amb altres mètodes per comparar-ne els resultats.


\section{Posició del Sol al cel vist des de l'habitatge} \label{sec: seccio_2}
En aquesta secció ens proposem trobar la posició del Sol des d'un punt determinat de la Terra durant tot l'any.\footnote{\label{nota: habitatge}Per fer els càlculs hem agafat les coordenades d'un habitatge de Sant Cugat del Vallès: 41°28'03.4"N 2°04'28.4"E} Per fer-ho hem parametritzat la posició del sol amb dos angles, l'angle vertical $\nu$ i l'angle horitzontal $\eta$ (\ref{fig: sist_sol}), i hem definit els vectors $\vec{\rho}$, del centre del Sol al punt de la superfície de la Terra, $\vec{r}$, del centre del Sol al punt de la superfície de la Terra, i $\vec{R}$, del centre del Sol al centre de la Terra (\ref{fig: sist_vectors}).
\begin{figure}[hbt]
    \centering
    \begin{subfigure}{0.5\textwidth}
        \centering
        \includegraphics[width=\textwidth]{vectors.PNG}
        \caption{Els vectors que hem definit a la secció \ref{sec: seccio_2}.}
        \label{fig: sist_vectors}
    \end{subfigure}%
    \hspace{0.000001\textwidth}%
    \begin{subfigure}{0.5\textwidth}
        \centering
        \includegraphics[width=\textwidth]{ang_sol.PNG}
        \caption{Els dos angles que hem usat per a determinar la posició del Sol.}
        \label{fig: sist_sol}
    \end{subfigure}
\end{figure}

També hem definit tres sistemes de referència (figura \ref{fig: sist_ref}) per facilitar els càlculs i trobar els àngles esmentats. Començant al sistema $\Omega$ podem obtenir un vector al sistema $\gamma$ a través de les matrius de rotació de l'annex \ref{annex: matr_rot}. Les relacions entre aquests sistemes són
\begin{itemize}
    \item Sistema $\Omega$: l'eix $z$ està orientat amb l'eix de rotació de la Terra.
    \item Sistema $\beta$: sistema rotat un angle $\beta$ en el pla $zy$ respecte el sistema $\Omega$. $\beta$ és l'angle entre l'eix de rotació de la terra i el vector perpendicular al pla de l'orbita, per tant, ara l'eix $z$ apunta en direcció pendicular al pla de l'òrbita.
    \item Sistema $\gamma$: sistema rotat un angle $\gamma$ en el pla $xy$ respecte el sistema $\beta$. $\gamma$ és l'angle entre $\vec{R}_{periheli}$ i $\vec{R}_{solstici(hivern)}$, per tant, ara la direcció de l'eix $y$ coincideix amb la direcció del periheli.
\end{itemize}
de tal manera que al sistema $\gamma$ tenim l'òrbita de la Terra amb el periheli i l'afeli sobre l'eix y i l'eix de rotació de la Terra orientat per tal que al solstici d'hivern (21 de Desembre) l'angle entre l'eix de rotació de la Terra i la direcció perpendicular al pla de l'òrbita sigui màxim.

Al sistema $\Omega$ el vector $\vec{r}$ és molt fàcil de definir 
\begin{equation}
    \vec{r}_{(t)}=r_T[\cos(\alpha)\cos(wt+\varphi)\hat{e_x}+\cos(\alpha)\sin(wt)+\varphi\hat{e_y}+\sin(\alpha)\hat{e_z}]
    \label{vector_r}
\end{equation}
si tenim la coordenada de latitud del punt de la Terra on volem fer els càlculs, $\alpha$, i el radi de la Terra, $r_T$, i on $w$ és la velocitat angular de rotació de la Terra i $t$ el temps transcorregut al llarg del dia.
Per tal de que aquest vector a $t=0$ comenci a la posició més allunyada del Sol (és a dir que t=0 es correspongui amb la meitat de la nit) hem d'afegir l'angle $\varphi$ a l'angle de rotació. $\varphi$ anirà variant cada dia i és l'angle entre el vector definit a (\ref{vector_r}) i el vector $\vec{R}$ a $t=0$.

El vector $\vec{R}$ ja el tenim calculat de la secció \ref{sec: seccio_1} i el vector $\vec{\rho}$ és simplement
\begin{equation}
    \vec{\rho}= \vec{R}+\vec{r}
\end{equation}
Un cop tenim aquests tres vectors els angles queden definits
\begin{equation}
    \nu=\angle (\vec{r}_{\gamma}, \vec{\rho}_{\gamma}) -\frac{\pi}{2}  
\end{equation}
\begin{equation}
    \eta=\pi - \angle (\vec{R}_{\Omega}, \vec{r}_{\Omega})
\end{equation}
on usem la funció arccosinus per trobar els angles.\footnote{$\angle (\vec{u}, \vec{v})= \arccos\left(\frac{\vec{u} \cdot \vec{v}}{\|\vec{u}\| \|\vec{v}\|}\right)$}


Si calculem aquests angles per una posició concreta a la Terra\footref{nota: habitatge}i pels dies que corresponen als equinocis i als solsticis obtenim el següent gràfic.
\begin{figure}[H]
    \centering
    \includegraphics[width=0.5\textwidth]{equinocis.png}
    \label{solsticis}
    \caption{Angles del Sol al llarg d'un dia per diferents moments de l'any a un habitatge de Sant Cugat del Vallès}
\end{figure}

\section{Estudi de l'energia elèctrica}
\subsection{Potència elèctrica produïda per la placa}
Suposant que el Sol és un cos negre, segons la llei de Stefan-Boltzmann l'intensitat de radiació emesa pel Sol és $I_S = \sigma_{SB}T_S^4$, on $T_S$ és la temperatura de la superfície del Sol. Així doncs, la potència total emesa ve donada per
\begin{equation}
    P_S = I_SA_S=\sigma_{SB}T_S^4 4 \pi R_S^2
    \label{potencia sol}
\end{equation}
on $A_S$ és l'àrea de la superfície del Sol, el qual hem considerat com una esfera de radi $R_S$.

La intensitat de radiació solar que arriba a un punt a una distància $d$ del Sol $I_d$ és igual la potència $P_S$ dividida entre l'àrea del front d'ona de la radiació, que és una closca esfèrica de radi $d$, considerant que la propagació és radial. Així doncs,
\begin{equation}
    I_d = \frac{P_S}{4\pi d^2}=\frac{1}{d^2}R_S^2\sigma_SB T_S^4
    \label{I_d}
\end{equation}
on a l'última igualtat hem substituit $P_S$ per l'Eq. \eqref{potencia sol}.

Substituint $d$ per la distància del Sol a un determinat punt sobre la superfície terrestre, $\rho$, a l'Eq. \eqref{I_d}, podem calcular la intensitat de radiació que hi incideix. Per considerar l’efecte de l’albedo terrestre $\alpha_A$, cal multiplicar aquesta intensitat per un factor $\alpha=1-\alpha_A$, que representa la fracció de radiació absorbida per la Terra. Així doncs, la intensitat de l'ona incident és $I_{abs}=\alpha I_{\rho}$.

Degut a la inclinació de la placa, la intensitat efectiva que rebrà serà igual a $I_{abs} \cos{\theta}$, on $\theta$ és l'angle d'incidència de la radiació, és a dir, l'angle entre la direcció de la llum incident i la normal a la superfície. Per tant, la potència que arriba a la placa ve donada per
\begin{equation}
    P_{inc} = A \alpha I_{\rho} \cos{\theta}
    \label{P_inc placa}
\end{equation}
on $A$ és l'àrea de la placa i hem substituit $I_{abs}$.

Finalment, la potència elèctrica generada per la placa, $P$, és igual a $P_{inc}$ multiplicada per un factor $r$, el rendiment de la placa. Així doncs, substituint l'Eq. \eqref{I_d} en l'Eq. \eqref{P_inc placa}, obtenim
\begin{equation}
    P = r \alpha \frac{1}{\rho^2}R_S^2\sigma_{SB}T_S^4A \cos{\theta}
    \label{potencia placa}
\end{equation}
Els valors numèrics de $\alpha$, $R_S$, $\sigma_{SB}$ i $T_S$ els hem obtingut de les fonts !!!! REFERENCIAR

A continuació donem la normalització de les variables de l'equació anterior. En quant a la variable $\theta$, al tractar-se d'un angle ja és adimensional, i per tant la donem per normalitzada. En quant a les altres dues variables,
\begin{align}
    \hat{P}=\frac{P}{400} \label{P normalizada} \\
    \hat{\rho}= \rho \left( r \alpha R_S^2\sigma_{SB}T_S^4A/400 \right)^{-1/2} \label{rho normalizada}
\end{align}
on hem escollit el valor 400 W per a normalitzar la potència perquè, segons l'enunciat, és la màxima electricitat que pot generar la placa per $10^3$ W/$\text{m}^2$ de radiació incident. També utilitzem aquests valors per a trobar el valor del rendiment $r$ de la placa de la següent manera:
\begin{equation}
     r = \frac{P_{max}}{IA}
\end{equation}
amb $P_{max}=400$ W i $I=10^3$ W/$\text{m}^2$.

Així doncs, segons aquesta normalització l'Eq. \eqref{potencia placa} esdevé
\begin{equation}
    \hat{P} = \frac{\cos{\theta}}{\hat{\rho}^2} \ .
    \label{pot norm}
\end{equation}
Per a determinar $\cos{\theta}$, utilitzem la següent expressió:
\begin{equation}
    \cos \theta = \cos \theta_z \cos \beta + \sin \theta_z \sin \beta \cos (\gamma_s - \gamma) \ .
    \label{cos theta}
\end{equation}
Les definicions dels angles i la demostració d'aquesta equació es troben a la secció \ref{sec: cos theta} de l'Annex.

A la Figura següent es representa la potència en funció del temps al llarg d'un dia, en dos dies diferents.

\begin{figure}[H]
    \centering
    \begin{subfigure}{0.5\textwidth}
        \centering
        %\includegraphics[width=\textwidth]{../3\)/dia1_pot_plot.png}
        \caption{Dia 1.}
        \label{fig: potencia dia 1}
    \end{subfigure}%
    \hspace{0.000001\textwidth}%
    \begin{subfigure}{0.5\textwidth}
        \centering
        %\includegraphics[width=\textwidth]{../3\)/dia170_pot_plot.png}
        \caption{Dia 170.}
        \label{fig: potencia dia 170}
    \end{subfigure}
    \caption{Potència elèctrica produïda per la placa durant un dia per diversos valors de $\beta$ i de $\gamma$.}
    \label{potencia un dia}
\end{figure}

Quan $\beta=0^{\circ}$, es representa la gràfica per un sol valor de $\gamma$ (en aquest cas, $\gamma=0^{\circ}$) perquè, segons l’Eq. \eqref{cos theta}, el terme $\sin\beta$ s’anul·la i, per tant, desapareix la dependència en $\gamma$.

També és important destacar que si $\theta_z$ o $\theta$ surten del rang $[-\pi/2,\pi/2]$ imposem que la potència sigui nul·la.

En el cas de $\theta_z$ ho imposem pel fet que per a valors fora d'aquest interval el Sol es troba per sota de l'horitzó i la Terra bloqueja la radiació solar, impedint que arribi llum a la placa. És per aquest motiu que, mentre que per $\beta = 0$ la potència disminueix gradualment fins a anul·lar-se, amb una inclinació diferent de zero la potència s'anul·la bruscament, ja que el Sol encara incideix amb una certa intensitat quan està a l’horitzó.

De manera similar, quan $\theta$ pren valors fora d'aquest interval significa que el Sol està situat darrere la placa, sent ella mateixa la que bloqueja la llum. És per això que en alguns casos, com per exemple per $\gamma=\pm60^{\circ}$ en la Fig. \ref{fig: potencia dia 1}, s'observa que la potència s'anul·la abans que el Sol creui l'horitzó.

En la Fig. \ref{fig: potencia dia 1} com més proper és el valor de $\gamma$ a 0$^{\circ}$, major és el valor màxim de la potència. D'altra banda, per un determinat $\gamma$, la potència màxima assolida quan $\beta=0^{\circ}$ és la més petita.

En el cas del dia 170, representat a la Fig. \ref{fig: potencia dia 170}, la potència màxima assolida per $\beta=0^{\circ}$ és notablement superior a la del dia 1. A més, per tots els casos en què $\beta=35^{\circ}$ la potència disminueix lleugerament al migdia. Aquests fets es deuen a que en ser el dia 170 d'estiu la posició del Sol és més alta en el cel, fet que afavoreix que quan el Sol estigui al voltant de la seva alçada màxima l’angle $\theta$ respecte a la placa sigui més gran per $\beta=0^{\circ}$ i disminueixi amb la inclinació de la placa. Una altra observació respecte el fet que es tracti d'un dia d'estiu és l’augment en la durada de les hores de llum.

\subsection{Energia elèctrica produïda per la placa}

Amb la potència elèctrica generada per la placa cada minut de l'any ens disposem a calcular l'energia elèctrica produïda per aquesta. La potència és la derivada temporal de l'energia, per tant, podem trobar l'energia elèctrica produïda per la placa durant una certa quantitat de temps integrant numèricament la potència respecte el temps. Per fer-ho, hem optat pel mètode d'integració numèrica de Simpson $\frac{1}{3}$. Presentem l'energia produïda cada dia de l'any per diferents combinacions dels angles $\beta$ (inclinació de la placa) i $\gamma$ (orientació de la placa) a la Fig. \ref{fig: energies}.


\begin{figure}[H]
    \centering
    \includegraphics[width=0.75\linewidth]{energia.png}
    \caption{Energia elèctrica produïda per la placa cada dia de l'any.}
    \label{fig: energies}
\end{figure}

Les linies negres verticals indiquen els solsticis d'estiu i d'hivern per a una millor interpretació dels resultats.

Pel que fa a la normalització del problema numèric, ens ha quedat la següent equació normalitzada.
\begin{equation}
    \hat{E} = \int \hat{P} \, d\hat{t}
    \label{energia}
\end{equation}
on $\hat{t}=\frac{t}{t_0}$, $\hat{P}=\frac{P}{P_0}$ i $\hat{E}=\frac{E}{P_0 t_0}$, amb $t_0=86400$ s (els segons que hi ha en un dia) i $P_0=400$ W (la potència màxima de la placa en les condicions donades a l'enunciat de la pràctica). 
\subsection{Extra: Optimització dels angles de la placa}

Hem plantejat el problema des del punt de vista de maximitzar la producció d'energia total al llarg d'un any mantenint els angles $\beta$ i $\gamma$ de la placa constants durant tot l'any. Per fer-ho, hem utilitzat que si tenim en compte l'Eq. \eqref{P normalitzada} i \eqref{cos theta}, l'energia total normalitzada produïda en un any la podem escriure com
\[
\hat{E}_T(\beta, \gamma) = \cos \beta \int \hat{P}_{opt} \cos \theta_{z} \, d\hat{t} + \sin \beta \cos \gamma \int \hat{P}_{opt} \sin \theta_{z} \cos \eta \, d\hat{t} + \sin \beta \sin \gamma \int \hat{P}_{opt} \sin \theta_{z} \sin \eta \, d\hat{t} \equiv
\]
\[
\equiv \hat{a}\cos \beta
+ \hat{b}\sin \beta \cos \gamma
+ \hat{c}\sin \beta \sin \gamma
\]
on $\hat{P}_{opt}=\frac{r I_{abs} A_T}{P_0}$, $\hat{t}=\frac{t}{t_0}$ i $\hat{E}=E P_0 t_0$

Igualant el gradent a 0, trobem que
\[
\left.
\begin{aligned}
-\hat{a}\sin \beta + \hat{b}\cos \beta \cos \gamma + \hat{c}\cos \beta \sin \gamma = 0 \\
-\hat{b}\sin \beta \sin \gamma + \hat{a}\sin \beta \cos \gamma = 0
\end{aligned}
\right\}
\]
La solució corresponent per $\beta$ i $\gamma$ és
\[
\gamma = \arctan\left(\frac{\hat{c}}{\hat{b}}\right), \quad \beta = \arctan\left(\frac{\hat{b} \cos(\gamma) + \hat{c} \sin(\gamma)}{\hat{a}}\right)
\]
on hem descartat la solució corresponent a $\beta=n\pi$ per no ser un màxim.

Calculant numèricament les integrals $\hat{a}$, $\hat{b}$ i $\hat{c}$ per Simpson 1/3 sota la condició de que $-\frac{\pi}{2} \leq \theta_z \leq \frac{\pi}{2}$ (en cas contrari l'integrand l'hem anul·lat), hem obtingut uns angles òptims arrodonint a la primera xifra decimal de $\gamma=0^\circ$ i $\beta=44,5^\circ$. L'energia produïda dia a dia durant tot l'any corresponent a aquests angles pot trobar-se a la Fig. \ref{fig: energies}. 

A la Fig. \ref{fig: energies} podem observar que a l'estiu un angle d'inclinació de $35^\circ$, per exemple, produeix més energia que un de $44,5^\circ$, ara bé, com hem dit, nosaltres hem maximitzat la producció total durant l'any. Això, però, suggereix que durant diferents èpoques de l'any podriem modificar l'inclinació de la placa per així maximitzar la seva producció en cada període i conseqüentment durant tot l'any.

\section{Resolució de l'EDO per diversos mètodes numèrics}\label{sec: edos}
Per a resoldre l'EDO de l'òrbita terrestre hem utilitzat el mètode d'Euler, però sabem que hi ha mètodes numèrics més potents que aquest. Per axò hem volgut comparar els resultats.

Els mètodes emprats per a resoldre l'equació \eqref{equ_en_r} han estat el Runge-Kutta d'ordre 2 i el Runge-Kutta d'ordre 4, amb la mateixa discretització temporal en els 3 mètodes numèrics.

\begin{figure}[hbt!]
    \centering
    \begin{subfigure}{0.5\textwidth}
        \centering
        \includegraphics[width=\textwidth]{orbitaRK2.PNG}
        \caption{Òrbita terrestre per Runge-Kutta d'ordre 2}
        \label{fig: orbitaRK2}
    \end{subfigure}%
    \vspace{0.01\textwidth}%
    \begin{subfigure}{0.5\textwidth}
        \centering
        \includegraphics[width=\textwidth]{orbitaRK4.PNG}
        \caption{Òrbita terrestre per Runge-Kutta d'ordre 4}
        \label{fig: orbitaRK4}
    \end{subfigure}
    \vspace{0.01\textwidth}%
    \begin{subfigure}{0.5\textwidth}
        \centering
        \includegraphics[width=\textwidth]{orbita3met.PNG}
        \caption{Òrbites terrestres per 3 mètodes numèrics}
        \label{fig: orbita3met}
    \end{subfigure}
    \caption{Càlcul òrbita terrestre per diferents mètodes numèrics}
    \label{fig: grafics3met}
\end{figure}

Observant els gràfics \ref{fig: grafics3met} podem veure que les diferències entre les òrbites càlculades pels diferents mètodes numèrics són realment mínimes.

\section{Consideració de la Terra com no esfèrica}\label{sec: terranoesfera}
Fins ara hem estat considerant que la Terra tenia forma d'una esfera perfecta. Però sabem que realment, la força centrífuga generada per la seva pròpia rotació provoca una deformació en els pols, "aixafant-la" i fent que la Terra no sigui una esfera.

Això no ens afecta al càlcul de l'òrbita terrestre ja que aquesta només considera la distància del sol al centre de la Terra però sí afecta en la resta de càlculs ja que el Radi de la Terra deixa de ser constant. En aquest apartat, aproximarem la Terra a un esferoide, com a un el·lipsoide de revolució\footnote{\label{nota: elipsoide}Els semieixos que hem agafat per a calcular el radi de la terra com a una esferoide són a = 6378136.6 i b = 6356751.9, extrets de l'article de Viquikipèdia: El·lipsoide de referència}

\begin{figure}[hbt]
    \centering
    \includegraphics[width=0.5\textwidth]{Terraelipsoide.jpg}
    \caption{Aproximació de la Terra a un el·lipsoide de revolució, amb els angles i semieixos que hem d'utilitzat per recalcular el Radi Terrestre i l'angle de Latitud}
    \label{fig:terraelipsoide}
\end{figure}

Observant la figura \ref{fig:terraelipsoide}, podem acabar arribant a les següents expressions:

\begin{equation}
    tan(\alpha_{T_{el}}) = \frac{b}{a}tan(\alpha) 
\end{equation}

\begin{equation}
    r_{T_{el}} = \sqrt{\frac{(a^2cos(\alpha))^2+(b^2sin(\alpha))^2}{(acos(\alpha))^2+(bsom(\alpha))^2}}
\end{equation}

Aquesta ens permet calcular el Radi de la nostra esferoide (distància superfície-centre terreste) en funció de l'angle de latitud $\alpha$, calculat ja a partir de la dades geogràfiques en la secció \ref{sec: seccio_2}. Redefinim $r_T$ com $r_{T_el}$ i $\alpha$ com $\alpha_{T_{el}}$ per poder calcular el nou vector $\vec{r_{(t)}}$ amb l'equació \eqref{dist_centresup}. Els passos per a la simulació són exactament els mateixos que els ja definits a les corresponents seccions \ref{sec: seccio_2} i \ref{sec: seccio_3}.

\section*{Annex}
\appendix

\section{Matrius de canvi de sistema de referència}\label{annex: matr_rot}
\begin{equation}
    \mathbf{R}_{\beta}=
    \begin{pmatrix}
      1 & 0 & 0   \\
      0 & \cos\beta& -sin\beta \\
      0 & sin\beta & cos\beta \\
    \end{pmatrix}
\end{equation}  

\begin{equation}
    \mathbf{R}_{\gamma}=
    \begin{pmatrix}
       \cos\beta& -sin\beta& 0 \\
       \sin\beta & cos\beta &0\\
      0 & 0 & 1  \\
    \end{pmatrix}
\end{equation}  


\section{Angles i vectors apartat 2}
\begin{figure}[hbt]
    \centering
    \begin{subfigure}{0.5\textwidth}
        \centering
        \includegraphics[width=\textwidth]{vectors.PNG}
        \caption{Els vectors que hem definit a la secció \ref{sec: seccio_2}.}
        \label{fig: sist_vectors}
    \end{subfigure}%
    \hspace{0.000001\textwidth}%
    \begin{subfigure}{0.5\textwidth}
        \centering
        \includegraphics[width=\textwidth]{ang_sol.PNG}
        \caption{Els dos angles que hem usat per a determinar la posició del Sol.}
        \label{fig: sist_sol}
    \end{subfigure}
\end{figure}

\begin{figure}[hbt]
    \centering
    \includegraphics[width=0.5\textwidth]{sist_ref.PNG}
    \caption{Diferents sistemes de referència de la secció \ref{sec: seccio_2}.}
    \label{fig: sist_ref}
\end{figure}
\end{document}
