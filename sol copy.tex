\documentclass[11pt]{article}
\usepackage{graphicx} % Required for inserting images
\usepackage[top=2.5cm, bottom=2.5cm, left=2.5cm, right=2.5cm]{geometry}
\usepackage[T1]{fontenc}
\usepackage{hyperref}
\usepackage[utf8]{inputenc}
\usepackage{multirow}
\usepackage{subcaption}
\usepackage{booktabs}
\usepackage{bookmark}
\usepackage{graphicx}
\usepackage{setspace}
\setlength{\parindent}{0in}
\usepackage{physics}
\usepackage{tikz}
\usepackage{tikz-3dplot}
\usepackage[outline]{contour} % glow around text
\usepackage{xcolor}
\usepackage{float}
\usepackage{makeidx}
\usepackage{fancyhdr}
\usepackage{pgfplots}
\usepackage{amsmath}
\pgfplotsset{compat=1.18}
\usepackage{caption}
\usepackage[english,catalan]{babel}
\setlength{\parskip}{11pt}
\usepackage{xcolor}
\usepackage{listings}
\usepackage{marginnote}

\title{\Huge\bfseries Pràctica de simulació: \\ Instal·lació de panells solars fotovoltaics en un habitatge unifamiliar a Catalunya \\ [2ex] \Large}

\author{\begin{tabular}{c}
\textbf{GRUP C3} \\
Isaac Baldi García (1667260)\\
Marcel López Freixes (1668323) \\
Eira Jacas García (1666616) \\
Núria Castillo Ariño (1669145)
\end{tabular}}

\date{07/1/01/2025}

\begin{document}

\maketitle

Abstract pràctica 

\section{Moviment de la Terra al voltant del Sol} 
En aquesta secció ens hem proposat simular el moviment de translació de la Terra al voltant del Sol. Per fer-ho hem partit de la Llei de la Gravitació Univeral i hem simplificant el nostre problema de dos cossos a un d'un sol cos sota una força central, $F(r)$.
\begin{equation}
    F(r)=-\frac{GMm}{r^2}
\end{equation}

On G és la constant de grabitació universal, M la massa del Sol i m la massa de la Terra.\footnote{Totes les dades orbitalàries agafades del Jet Propulsion Laboratory de la NASA: \url{https://ssd.jpl.nasa.gov/}}

Per aquest tipus de sistemes i considerant únicament aquesta força central, tenim dues equacions de moviment en el pla polar 
\begin{equation}
    F(r)=m\ddot{r}-mr{\dot{\theta}}^2
    \label{equ_en_r}
\end{equation}
\begin{equation}
    0=\ddot{\theta}m=mr\ddot{\theta}+2m\dot{r}\dot{\theta}
    \label{equ_en_theta}
\end{equation}
i la propietat que el moment angular es conserva
\begin{equation}
    L=mr\dot{\theta}=ctt
    \label{moment_angular}
\end{equation}

Combinant les equacions \eqref{equ_en_r} i \eqref{moment_angular} obtenim una EDO que només depen de r i una EDO que només depen de $\theta$
\begin{equation}
    \frac{\partial\dot{r}}{\partial t}=-GM\frac{1}{r^2}+\frac{L^2}{m^2r^3}
    \label{edor}
\end{equation}
\begin{equation}
    \frac{\partial}{}
    \label{edot}
\end{equation}
Normalitzant aquestes dues equacions i reduint l'ordre de l'equació \eqref{edor} obtenim
\begin{equation}
    \frac{\partial\tilde{v}}{\partial\tilde{t}}=-\frac{1}{\tilde{r}^2}+\frac{1}{\tilde{r}^3}
    \label{1_edo_r}
\end{equation}
\begin{equation}
    \frac{\partial\tilde{r}}{\partial\tilde{t}}=\tilde{v}
    \label{2_edo_r}
\end{equation}
\begin{equation}
    \frac{\partial\tilde{\theta}}{\partial\tilde{t}}=\frac{1}{\tilde{r}^2}
    \label{edo_tetha}
\end{equation}
on les variables normalitzades segueixen $r=\tilde{r}\alpha$, $t=\tilde{t}\frac{\alpha}{\bar{v}}$, $v=\tilde{v}\bar{v}$ i les constants de normalització $\alpha = \frac{\beta}{\kappa}$, $\bar{v}=\frac{\kappa}{(\beta)^{1/2}}$, $\beta=\frac{L^2}{m^2}, \kappa=GM$.

Aquest sistema d'equacions diferencials de primer ordre l'hem resolt numèricament amb el mètode d'Euler i agafant com a condicions de contorn el radi de l'òrbita, la velocitat radial i l'angle al periheli.\footnotemark[\value{footnote}]



\section{Posició del Sol al cel vist des de l'habitatge} 

\section{Estudi de l'energia elèctrica}

\section{Resolució de l'EDO per diversos mètodes numèrics}

\subsection{Resolució de l'EDO: Runge-Kutta d'ordre 2}

\subsection{Resolució de l'EDO: Runge-Kutta d'ordre 4}

\end{document}